\documentclass[11pt,a4paper]{article}
\usepackage[utf8]{inputenc}
\usepackage[T1]{fontenc}
\usepackage[catalan]{babel}

\title{Resum article The Death of Big Software \\(CACM December 2017, 60:12 pp 29-32)}
\date{Enginyeria del Software II 2023/24}
\author{Aniol Juanola Vilalta, Guillem Vidal Serrats, Jordi Badia Auladell}
\begin{document}
\maketitle

L’article The Death of Big Software ens parla sobre la mort de softwares “grans” en tots els àmbits des dels anys 90. 

Les causes de mort s’atribueixen a quatre factors, entre els quals es troba la necessitat de controlar els processos intraempresarials i posar ordre en la diversitat d’aplicacions casolanes que usaven. La solució no va millorar el problema, si no que resultà en la nova manca de control per part dels venedors degut a l’extens poder del software.

Similarment, l'estandardització i normalització del software va aplicar una rigidesa sobre moltes aplicacions que s’haurien desenvolupat diferentment i haurien aportat solucions diferents a problemes similars, resultant en la mort de molts projectes altrament fruitosos.

%fins aqui

Abans, les empreses desenvolupaven els sistemes a dins dels seus propis servidors. És a dir, no només es dedicaven a construir tota l'arquitectura, sinó que també la logística d'on s'executarien aquests programes. Això causava retrassos degut a la complexitat de sincronitzar i gestionar masses equips, nivells d'abstracció i funcionalitats d'un mateix sistema en un lloc centralitzat. Per culpa d'això, els projectes se n'anaven molt per sobre del pressupost i els resultats sempre deixaven molt a desitjar.

El núvol va sorgir com una alternativa que permetia alleujar els problemes d'implementació d'aquests sistemes, delegant la gestió dels servidors a empreses alienes.

Ara bé, potser un dels problemes més importants en el Big Software eren els arquitectes, diu l'article. Perquè molt sovint, en paper es dissenyava un sistema tan complexe que ni els mànagers més experimentats tenien la capacitat de gestionar tals projectes. A més a més, el disseny monolític de tals arquitectures, tot i semblar més rígides i inflexibles, feia impossible una gestió fragmentada, simplificada o independent dels projectes.

%fins aqui

Per tal de combatre aquestes tares, els arquitectes proposen arquitectures basades en microserveis: aplicacions conformades de molts petits serveis que poden ser desenvolupades i mantingudes independentment, i que es comuniquen entre ells mitjançant mecanismes més lleugers sense estructura centralitzada.

Aquesta tecnologia combinada amb l'arquitectura de contenidors (paquets de software que inclouen tots el elements necessaris per executar el programa en qualsevol entorn) podrien resultar en la resposta a la arquitectura monolítica.
Així doncs, aquells que ara son els assassins del “Big Software”, son els mateixos que directa o indirectament ens brindaran el que podríem anomenar “Small Software”.
\end{document}